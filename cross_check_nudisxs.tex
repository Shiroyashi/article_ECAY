\subsection{Валидация \texttt{nudisxs}}

Для проверки корректности расчётов полный спектр сечений, полученных с помощью \texttt{nudisxs}, был сопоставлен с экспериментальными данными по глубоконеупругому рассеянию нейтрино и антинейтрино на нуклонах. 
На рис.~\ref{fig:disxs_compare} показано сравнение предсказаний, полученных с помощью~\texttt{XsDis}~\cite{kuzmin2006_finetuning,kuzmin2005_sumcc,kuzmin2006_axialmass} (ядро вычислений \texttt{nudisxs}), с данными различных экспериментов, охватывающих диапазон энергий от единиц ГэВ до~$10^6$~ГэВ.

Наблюдается хорошее согласие с экспериментальными данными. Более детальный анализ приведён в работах~\cite{kuzmin2006_finetuning,kuzmin2005_sumcc,kuzmin2006_axialmass}.

\begin{figure}[!h]
\centering
\includegraphics[width=\linewidth]{images/dis_vs_data.pdf}
\caption{Сравнение полного сечения глубоконеупругого рассеяния нейтрино и антинейтрино, рассчитанного с помощью \texttt{XsDis}~\cite{kuzmin2006_finetuning,kuzmin2005_sumcc,kuzmin2006_axialmass}.}
\label{fig:disxs_compare}
\end{figure}

