\section{Intro}
Современная многоканальная астрономия основывается на объединении данных из различных наблюдательных каналов — электромагнитного излучения (от радиодиапазона до гамма-квантов), гравитационных волн, космических лучей и нейтрино. Такой подход позволяет формировать целостное представление об астрофизических процессах, которое недостижимо при использовании лишь одного вида излучения.

Космические лучи, представляющие собой протоны и атомные ядра, служат инструментом для изучения высокоэнергетических процессов во Вселенной уже более ста лет. Важнейшую роль в этом направлении играют космические лучи сверхвысоких энергий (КЛСВЭ) — частицы с энергией выше $10^9$~ГэВ. Они потенциально связаны с экстремальными астрофизическими объектами и событиями, включая активные ядра галактик, пульсары, сверхновые вспышки~\cite{auger2020anisotropy, auger2020spectrum}, гамма-всплески~\cite{kotera2011astrophysics} и слияния нейтронных звёзд~\cite{kimura2017ultrahigh}. Изучение таких частиц не только углубляет понимание механизмов ускорения до колоссальных энергий, но и открывает возможности для поиска новых физических явлений за пределами Стандартной модели.

Тем не менее, информативность КЛСВЭ ограничивается тремя ключевыми факторами. Во-первых, заряженные частицы отклоняются магнитными полями, что препятствует точной локализации источников. Во-вторых, протоны с энергиями выше $5\times10^{10}$~ГэВ теряют энергию в результате взаимодействия с космическим микроволновым фоном (эффект Грейзена–Зацепина–Кузьмина~\cite{greisen1966}). В-третьих, поток КЛСВЭ чрезвычайно мал по сравнению с потоком нейтрино меньших энергий, что требует детекторов гигантского объёма.

Гамма-астрономия охватывает широкий диапазон энергий фотонов (от 0.1 Мэв до 1 ПэВ), позволяя изучать релятивистские струи, аккреционные диски и пульсары. Современные наземные обсерватории, такие как H.E.S.S.~\cite{hess2021}, MAGIC~\cite{hessandmagic2021}, TAIGA~\cite{Elshoukrofy:2023My} и другие, успешно регистрируют гамма-кванты с энергией до $10^5$~ГэВ. Однако наблюдения в этом диапазоне ограничены межгалактическим поглощением на фоне инфракрасного и микроволнового излучений из-за процессов $\gamma\gamma \to e^+e^-$.

Становление гравитационно-волновой астрономии ознаменовалось регистрацией сигналов от слияний компактных объектов детекторами LIGO и Virgo~\cite{virgoandligo2016}. Эти наблюдения предоставляют уникальные данные о двойных системах и взрывах сверхновых ~\cite{Abbott:2017, Fan:2024}. Преимуществом гравитационных волн является их способность распространяться сквозь вещество без значительных искажений~\cite{Isaacson1968}. Однако эффективность метода ограничивается амплитудой сигнала, которая может оказаться ниже порога чувствительности~\cite{LIGOScientific:2018Sens}.

Особое место среди астрономических инструментов занимают нейтрино сверхвысоких энергий. Благодаря своей нейтральности и крайне слабому взаимодействию с веществом, нейтрино способны достигать детекторов, сохраняя информацию об источнике, даже при прохождении через всю Землю. Эти свойства делают их не только мощным каналом многоканальной астрономии, но и возможным зондом для исследования внутренней структуры планеты.

Важнейшим аспектом интерпретации нейтринных данных является учёт взаимодействия частиц с веществом Земли. С одной стороны, Земля служит поглощающим экраном, ослабляющим нейтринные потоки. С другой — именно это ослабление несёт информацию о плотности вещества и сечениях взаимодействия, открывая возможности для нейтринной томографии.

Современные нейтринные телескопы, такие как IceCube, KM3NeT, Baikal-GVD и другие, располагают эффективными объёмами порядка $0.1$–$1~\text{км}^3$ и регистрируют черенковское излучение, возникающее при взаимодействии нейтрино в прозрачных средах~\cite{Troitskii:2024}. Для моделирования ожидаемых энергетических и угловых спектров требуется точное знание профиля плотности Земли (например, модели PREM), сечений нейтрино-нуклонного взаимодействия в области глубоконеупругого рассеяния, резонансное взаимодействие электронного антинейтрино с электроном, а также учёт процессов регенерации и затухания потока в веществе.

Существующие коллаборации применяют разнообразные подходы к моделированию. В IceCube используются пакеты \texttt{PROPOSAL}~\cite{Koehne:2013gpa} и \texttt{nuSQuIDS}~\cite{ARGUELLES2022108346}, решающие кинетические уравнения с учетом осцилляций нейтрино. KM3NeT реализует собственную  цепочку моделирования, где этапы генерации и распространения разделены~\cite{ARGUELLES2022108346}. 

В рамках настоящей работы разработаны два новых инструмента, предназначенных в первую очередь для моделирования в эксперименте Baikal-GVD, но свободно доступных для всего научного сообщества через платформу \texttt{PyPI}. Пакет \texttt{nudisxs} реализует вычисление дважды дифференциальных сечений глубоконеупругого рассеяния (DIS) с использованием партонных распределений из библиотеки \texttt{LHAPDF6}~\cite{Buckley_2015}. Пакет \texttt{NuPropagator} основан на методе $\mathcal{Z}$-фактора~\cite{Naumov:1998sf} и предназначен для моделирования эволюции нейтринного спектра при прохождении сквозь вещество. Оба инструмента легко интегрируются в существующие симуляционные цепочки и ориентированы на широкое применение в задачах нейтринной астрофизики.

Целями настоящей статьи являются: анализ достоверности глубоконеупругого сечения на ПэВ-ных энергиях с учётом вклада неизмеренного фазового пространства; демонстрация чувствительности нейтринных потоков к структуре земного ядра; описание реализации метода $\mathcal{Z}$-фактора и применение его в задачах моделирования; а также представление и тестирование разработанных инструментов. 