\section{Программный пакет \texttt{NuPropagator}}

\subsection{Общее описание и структура}

Пакет \texttt{NuPropagator}~\cite{nupropagator2022} представляет собой модуль для моделирования прохождения потоков нейтрино через вещество, в частности через Землю, с учётом взаимодействий по заряженному и нейтральному токам. Он реализует итеративный метод на основе $\mathcal{Z}$-фактора, что позволяет учитывать регенерацию нейтрино при рассеянии на нуклонах. Пакет написан на языке \texttt{Python3} и поддерживается через платформу \texttt{PyPI}, что обеспечивает его доступность и простоту интеграции в существующие симуляционные цепочки.

\begin{figure}[!h]
\centering
\includegraphics[width=\linewidth]{images/nupropagator_diagram.png}
\caption{Структура программного пакета \texttt{nupropagator} и его зависимости.}
\label{fig:nupropagator1}
\end{figure}

\subsection{Физическая модель}

Пакет \texttt{NuPropagator} использует следующие физические компоненты:
\begin{itemize}
  \item модель плотности Земли (PREM);
  \item сечения взаимодействия нейтрино с нуклонами, предоставляемые пакетом \texttt{nudisxs};
  \item итерационный метод Z-фактора для расчёта эволюции нейтринного спектра.
\end{itemize}

\subsection{Модель плотности Земли и расчёт толщины}

В качестве модели плотности используется Предварительная эталонная модель Земли (PREM)~\cite{dziewonskiPREM1981}, которая предполагает сферическую симметрию и описывает плотность, давление и другие параметры как функции радиуса. Плотность в модели представлена на рис.~\ref{PREM}.

\begin{figure}[!h]
\centering
\includegraphics[width=\linewidth]{images/NuProp/PREM.pdf}
\caption{Плотность вещества в модели Земли PREM.}
\label{PREM}
\end{figure}

Расчёт толщины вещества, проходимого нейтрино, осуществляется интегрированием плотности вдоль траектории. Поддерживаются несколько численных методов: метод прямоугольников, квадратурный метод \texttt{quad} из \texttt{SciPy}, и метод Монте-Карло \texttt{vegas}. Изотопный состав вещества задаётся в модуле Earth пакета \texttt{NuPropagator}.

\subsection{Метод Z-фактора}

Z-фактор представляет собой поправку к затуханию потока нейтрино, учитывающую регенерацию за счёт нейтрального тока. Метод подробно описан в работе~\cite{naumov1999} и базируется на решении следующего уравнения переноса нейтрино:
\begin{equation}
\frac{\partial F_{\nu}(x,E)}{\partial x} = \frac{1}{\lambda_{\nu}(E)}\left[ \int\limits_0^1\frac{dy}{1-y}\Phi_{\nu}(y,E) F_{\nu}(x,E_y) - F_{\nu}(x,E) \right],
\end{equation}
где $F_{\nu}(x,E)$ — поток нейтрино после прохождения толщины $x$, $E_y = E/(1-y)$, $\lambda{E}$ - полное сечение взаимодействия нейтрино с веществом, а $\Phi_{\nu}(y,E)$ — распределение по передаче энергии, определяемое следующей формулой:
\begin{equation}
    \Phi_{\nu}(y,E) = \frac{\sum\limits_{T\in \{n,p,e\}}N_T\frac{d\sigma_{\nu T}}{dy}(y,E_y)}{\sum\limits_{T\in \{n,p,e\}}N_T\sigma_{\nu T}(E)}
\end{equation}

Предполагаемое решение имеет вид:
\begin{equation}
F_{\nu}(x,E) = F^{0}_{\nu}(E)\exp\left(-\frac{x}{\Lambda_{\nu}(x,E)}\right),
\end{equation}
где 
\begin{equation}
\Lambda_{\nu}(x,E) = \frac{\lambda_{\nu}(E)}{1 - \mathcal{Z}_{\nu}(x,E)}.
\end{equation}

\subsection{Итерационный метод и его реализация}

Решение уравнения для $\mathcal{Z}_{\nu}(x,E)$ реализуется итерационно, начиная с $\mathcal{Z}^{(0)}$ и строя следующие приближения по схеме:
\begin{equation}
\mathcal{Z}^{(n+1)}_{\nu}(x,E) = \int\limits_0^x dx' \int\limits_0^1 dy\,\eta_{\nu}(y,E)\Phi_{\nu}(y,E)\exp\left[ -x'D^{(n)}_{\nu}(x',E,E_y) \right],
\end{equation}
где $\eta_{\nu}(y,E)$ — весовой фактор, определяемый формой начального спектра. Точное выражение для $\eta_{\nu}(y,E)$ имеет следующий вид: 
\begin{equation}
    \eta_{\nu}(y,E) = \frac{F^0_{\nu}(E_y)}{(1-y)F^0_{\nu}(E)}.
\end{equation}
Выражение для фактора $D^{(n)}_{\nu}(x, E, E_y)$ имеет следующий вид:
\begin{equation}
    D^{(n)}_{\nu}(x, E, E_y) = \frac{1-\mathcal{Z}_{\nu}^{(n)}(x, E_y)}{\lambda(E_y)} - \frac{1-\mathcal{Z}_{\nu}^{(n)}(x, E)}{\lambda(E)}
\end{equation}
Реализация учитывает как исчезновение нейтрино за счёт заряженного тока, так и эффект регенерации от нейтрального тока. Конечная плотность потока описывается экспоненциальным затуханием с учётом эффективной длины пробега:
\begin{equation}
P(E,x) = \exp(-x/\lambda_{\nu}(E)), \quad x = \int\rho(l)\,dl.
\end{equation}
\subsection{Сравнение с другими программными пакетами}
 Проведем сравнение результатов, полученных с помощью двух програмнных пакетов: \texttt{nuFATE}~\cite{Vincent_2017} и \texttt{nupropagator}. Будем сравнивать потоки мюонного нейтрино, приходящих под разными углами после прохождения через землю в некоторой точке, находящейся на глубине 1 км от поверхности Земли. В качестве нейтринного потока на поверхности земли был использован степенной поток $E^{-2}$. Также построим переменную $\delta_{asymm}$ в зависимости от энергии, которая определяется следующим образом: 
 \begin{equation}
     \delta_{asym} = 2\frac{F^{nupropagator}_{\nu}(E,x) - F^{nuFATE}_{\nu}(E,x)}{F^{nupropagator}_{\nu}(E,x) +
     F^{nuFATE}_{\nu}(E,x)}
 \end{equation}
\begin{figure}[!h]
\centering
\includegraphics[width=\linewidth]{images/NuProp/compNuandNu.pdf}
\caption{Сравнение потоков, полученных с помощью \texttt{nuFATE} и \texttt{Nupropagator} для разных углов прилета нейтрино.}
\label{fig:flux_compare}
\end{figure}
Как можно видеть из рис.~\ref{fig:flux_compare} результаты находятся в хорошем согласии при энергиях меньше $10^6$ГэВ. Дальнейшее расхождение можно объяснить как отсутствием учета эффекта регенерации потоков за счет таонного нейтрино, различным поведением моделей сечений при больших энергиях, различными численными способами решения кинетических уравнений, которым описывается эволюция потоков нейтрино.   



\subsection{Дополнительные возможности}
% (Оставьте место для дополнения при необходимости)

\subsection{Интеграция с другими модулями}
% (Добавьте описание связей с другими компонентами симуляционного фреймворка)

\subsection{Поддержка и установка}

Пакет \texttt{NuPropagator} доступен через \texttt{PyPI} и может быть установлен командой:
\begin{verbatim}
pip install nupropagator
\end{verbatim}
Полная документация доступна на странице проекта, включающей примеры использования и описание API.
