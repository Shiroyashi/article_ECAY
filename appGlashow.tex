\subsection{Сечение взаимодействия $\overline{\nu}_e e$}

Сечение взаимодействия нейтрино с электроном на три порядка меньше, чем с нуклоном, поэтому из взаимодействий с электроном в нашей работе учитывается только резонанс Глэшоу ($\bar{\nu}_e + e^- \to W^+$). Дифференциальные сечения, используемые в расчётах, приведены по работе~\cite{GANDHI199681}.

Сечение реакции $\bar{\nu}_e e \to \bar{\nu}_e e$:
\begin{equation}
\begin{aligned}
&\frac{d\sigma(\bar{\nu}_e e \to \bar{\nu}_e e)}{dy} 
= \frac{G_F^2 m_e E_\nu}{2\pi} 
    \left[ 
      \frac{R_e^2}{\left(1 + 2m_e E_\nu y / M_Z^2\right)^2} 
    \right] \\
&+ \frac{G_F^2 m_e E_\nu}{2\pi}
    \left[
      \left|
        \frac{L_e}{1 + 2m_e E_\nu y / M_Z^2}
        + \frac{2}{1 - 2m_e E_\nu / M_W^2 + i\,\Gamma_W / M_W}
      \right|^2 (1 - y)^2
    \right],
\end{aligned}
\end{equation}

Сечение реакции $\bar{\nu}_e e \to \bar{\nu}_\mu \mu$:
\begin{equation}
\frac{d\sigma(\bar{\nu}_e e \to \bar{\nu}_\mu \mu)}{dy} 
= \frac{G_F^2 m_e E_\nu}{2\pi} 
   \frac{
     4(1 - y)^2 \left[ 1 - (\mu^2 - m_e^2)/(2 m_e E_\nu) \right]^2
   }{
     \left(1 - 2m_e E_\nu / M_W^2\right)^2 + \Gamma_W^2 / M_W^2
   }.
\end{equation}

Сечение реакции $\bar{\nu}_e e \to \text{адроны})$:
\begin{equation}
\frac{d\sigma(\bar{\nu}_e e \to \text{адроны})}{dy} =
\frac{d\sigma(\bar{\nu}_e e \to \bar{\nu}_\mu \mu)}{dy}
\frac{\Gamma(W \to \text{адроны})}{\Gamma(W \to \mu \bar{\nu}_\mu)},
\end{equation}
где $\Gamma(W \to \text{адроны})$ и $\Gamma(W \to \mu \bar{\nu}_\mu)$ -- ширины распадов $W$ бозона на адроны и $\mu \bar{\nu}_\mu$.