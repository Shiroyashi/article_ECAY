\section{Введение}

Современная многоканальная астрономия объединяет данные из различных наблюдательных каналов — электромагнитного излучения, гравитационных волн, космических лучей и нейтрино, что позволяет получать целостное представление об астрофизических процессах, недостижимое в рамках отдельных наблюдений.

Космические лучи сверхвысоких энергий (КЛСВЭ, $E > 10^9$~ГэВ) служат важным инструментом для изучения экстремальных астрофизических источников — активных ядер галактик, пульсаров, сверхновых вспышек, гамма-всплесков и слияний нейтронных звёзд~\cite{auger2020anisotropy, auger2020spectrum, kotera2011astrophysics, kimura2017ultrahigh}.  
Однако их исследование осложнено: заряженные частицы отклоняются магнитными полями, протоны теряют энергию на космическом микроволновом фоне (эффект Грейзена–Зацепина–Кузьмина~\cite{greisen1966}), а интенсивность потока чрезвычайно мала.  

Гамма-астрономия, достигшая энергий до $10^5$~ГэВ благодаря экспериментам H.E.S.S.~\cite{hess2021}, MAGIC~\cite{hessandmagic2021} и TAIGA~\cite{Elshoukrofy:2023My}, ограничена межгалактическим поглощением фотонов на фоне инфракрасного и микроволнового излучения.  
Гравитационно-волновая астрономия (LIGO, Virgo~\cite{virgoandligo2016, Abbott:2017, Fan:2024}) открыла новый канал наблюдений, но её эффективность определяется амплитудой сигнала, часто ниже порога чувствительности~\cite{Isaacson1968, LIGOScientific:2018Sens}.  

На этом фоне нейтрино сверхвысоких энергий занимают особое место. Благодаря своей нейтральности и крайне слабому взаимодействию, нейтрино несут ненарушенную информацию об источнике, проходя через космическое пространство и толщу Земли. Это делает их не только уникальным инструментом многоканальной астрономии, но и потенциальным зондом внутренней структуры планеты.

Ключевым элементом интерпретации нейтринных данных является моделирование взаимодействия нейтрино с веществом Земли. С одной стороны, Земля выступает поглощающим экраном, ослабляющим поток, а с другой — именно это ослабление содержит информацию о плотности и составе вещества, открывая возможность нейтринной томографии.  
Современные нейтринные телескопы (IceCube, KM3NeT, Baikal-GVD) обладают эффективными объёмами $0.1$–$1~\text{км}^3$ и регистрируют черенковское излучение, возникающее при взаимодействии нейтрино в прозрачных средах~\cite{Troitskii:2024}. Для корректного моделирования их отклика необходимо надёжное знание профиля плотности Земли (например, модели PREM), сечений глубоконеупругого рассеяния нейтрино, а также учёт процессов регенерации потока.

Существующие коллаборации используют различные подходы: в IceCube применяются пакеты \texttt{PROPOSAL}~\cite{Koehne:2013gpa} и \texttt{nuSQuIDS}~\cite{ARGUELLES2022108346}, решающие уравнения переноса с учётом осцилляций, в KM3NeT реализована собственная цепочка генерации и распространения нейтрино~\cite{ARGUELLES2022108346}.  

В настоящей работе разработаны два открытых инструмента, ориентированные на моделирование потоков нейтрино высоких энергий в экспериментах Baikal-GVD и аналогичных установках:  
\texttt{nudisxs} — пакет для вычисления дважды дифференциальных сечений глубоконеупругого взаимодействия с использованием партонных распределений из библиотеки \texttt{LHAPDF6}~\cite{Buckley_2015},  
и \texttt{NuPropagator} — программа, реализующая $\mathcal{Z}$-факторный метод~\cite{Naumov:1998sf} для моделирования эволюции нейтринных потоков при прохождении через вещество.  
Оба инструмента интегрируются в существующие симуляционные цепочки и доступны через платформу \texttt{PyPI}.  

Целями работы являются:
\begin{itemize}
    \item оценка достоверности партонной модели при ПэВ-энергиях, включая вклад неисследованной области малых $x$;
    \item моделирование распространения нейтрино через Землю с учётом регенерации;
    \item анализ чувствительности нейтринной томографии к профилю плотности планеты;
    \item представление и тестирование разработанных программных пакетов.
\end{itemize}

Структура статьи следующая.  
Раздел~\ref{sec:dis} описывает кинематику и сечения глубоконеупругого взаимодействия нейтрино с нуклоном.  
Раздел~\ref{sec:nudisxs} посвящён программному пакету \texttt{nudisxs} и его валидации.  
В разделе~\ref{sec:dis_reliability} анализируется достоверность партонной модели на ПэВ-уровне энергий.  
В разделе~\ref{sec:zfactor} описан $\mathcal{Z}$-факторный метод решения транспортного уравнения для потоков нейтрино с учётом регенерации и рассчитана непрозрачность Земли.  
Раздел~\ref{sec:nupropagator} посвящён пакету \texttt{NuPropagator} и сравнению его результатов с альтернативными решениями.  
В разделе~\ref{sec:tomography} обсуждается нейтринная томография Земли и влияние регенерации, а в заключении~\ref{sec:conclusions} сформулированы основные выводы.
