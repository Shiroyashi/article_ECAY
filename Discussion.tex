\section{Обсуждение и выводы}
\label{sec:conclusions}
В настоящей работе представлены два открытых программных инструмента — \texttt{nudisxs} и \texttt{NuPropagator}, предназначенные для моделирования взаимодействий и распространения нейтрино высоких энергий. 
Первый пакет обеспечивает точное вычисление сечений глубоконеупругого рассеяния на основе современных партонных распределений, 
второй реализует $\mathcal{Z}$-факторный метод, позволяющий учитывать регенерацию нейтрино при прохождении через вещество. 
Оба инструмента ориентированы на применение в нейтринной астрофизике и моделировании детекторов типа IceCube, KM3NeT и Baikal-GVD.

Проведён анализ достоверности партонной модели в диапазоне энергий до $E_\nu \sim 10^9$~ГэВ. 
Показано, что основная неопределённость связана с отсутствием экспериментальных данных в области малых $x \lesssim 10^{-6}$, однако вклад этой области в полное сечение не превышает нескольких процентов. 
Сравнение различных наборов партонных распределений (\texttt{CT10nlo}, \texttt{CT18ZNNLO}, \texttt{nCTEQ15}, \texttt{TUJU19\_nlo}) демонстрирует согласие на уровне $\sim5\%$, что подтверждает надёжность предсказаний в пределах современных феноменологических моделей. 
Таким образом, расчёты нейтринных сечений вплоть до 100 ПэВ энергий можно считать устойчивыми и применимыми для задач астрофизики и нейтринной томографии.

Показано, что учёт регенерации, обусловленной взаимодействиями по нейтральному току, существенно усиливает чувствительность метода нейтринной томографии Земли. 
Без этого эффекта поток нейтрино экспоненциально затухает, тогда как включение регенерации сохраняет измеримый сигнал даже при энергиях $E_\nu \gtrsim 1$~ПэВ. 
Регенерация повышает статистическую значимость наблюдений и открывает возможность реконструкции плотностного профиля ядра по энергетическим спектрам проходящих нейтрино.

Полученные результаты демонстрируют, что крупномасштабные нейтринные телескопы нового поколения (IceCube-Gen2, KM3NeT, Baikal-GVD) смогут использовать потоки астрофизических нейтрино не только как инструмент астрономических наблюдений, но и как физический зонд внутренней структуры Земли. 
Дальнейшее развитие представленных подходов создаёт основу для объединённого анализа данных различных нейтринных обсерваторий, уточнения модели плотности и химического состава земного ядра, а также проверки фундаментальных свойств нейтрино при экстремальных энергиях.
