\subsection{Взаимодействие нейтрино с электроном}
Так как сечение взаимодействия нейтрино с электроном много меньше нуклонов, то в дальнейшем будет учитываться только резонанс Глешоу ($\bar{\nu}_e+e^{-}\to W^{+}$). Для расчета будем использовать следующие формулы для дифференциальных сечений, взятые из работы~\cite{GANDHI199681}:
\begin{equation}
    \begin{aligned}
\frac{d\sigma(\bar{\nu}_e e \rightarrow \bar{\nu}_e e)}{dy} 
&= \frac{G_F^2 m E_\nu}{2\pi} \left[ \frac{R_e^2}{\left( 1 + 2m E_\nu y / M_Z^2 \right)^2}\right]\\ 
&+ \frac{G_F^2 m E_\nu}{2\pi}\left[\left| \frac{L_e}{1 + 2m E_\nu y / M_Z^2} + \frac{2}{1 - 2m E_\nu / M_W^2 + i F_w / M_w} \right|^2 (1 - y)^2 \right],\\
\frac{d\sigma(\bar{\nu}_e e \rightarrow \bar{\nu}_\mu \mu)}{dy} 
&= \frac{G_F^2 m E_\nu}{2\pi} \frac{4(1-y)^2 [1-(\mu^2-m^2)/2mE_\nu]^2}{(1-2mE_\nu/M_W^2)^2 + \Gamma_W^2/M_W^2}, \\
\frac{d\sigma(\bar{\nu}_e e \rightarrow \text{hadrons})}{dy} 
&= \frac{d\sigma(\bar{\nu}_e e \rightarrow \bar{\nu}_\mu \mu)}{dy} \frac{\Gamma(W \rightarrow \text{hadrons})}{\Gamma(W \rightarrow \mu \bar{\nu}_\mu)},
\end{aligned}
\end{equation}
